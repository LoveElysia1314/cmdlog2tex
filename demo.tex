% Use ctexart document class for Chinese support
\documentclass{ctexart}
\usepackage[margin=1in]{geometry}
\usepackage{xcolor}
\usepackage[most]{tcolorbox}
\usepackage{listings}

% ============================================================================
% 统一终端环境定义 - 支持有色/无色模式切换
% ============================================================================

% ===== 颜色定义 =====
\definecolor{terminalbg}{RGB}{40,44,52}
\definecolor{terminalfg}{RGB}{171,178,191}
\definecolor{promptgreen}{RGB}{39,201,63}
\definecolor{pathblue}{RGB}{80,150,255}

% ANSI 颜色定义(用于有色模式)
\definecolor{ansiblack}{RGB}{0,0,0}
\definecolor{ansired}{RGB}{205,49,49}
\definecolor{ansigreen}{RGB}{0,255,0}
\definecolor{ansiyellow}{RGB}{255,255,0}
\definecolor{ansiblue}{RGB}{0,0,255}
\definecolor{ansimagenta}{RGB}{255,0,255}
\definecolor{ansicyan}{RGB}{0,255,255}
\definecolor{ansiwhite}{RGB}{255,255,255}
\definecolor{ansigray}{RGB}{128,128,128}
\definecolor{lime}{RGB}{0,255,0}



% ===== Listings 样式(用于无色模式)=====
\lstdefinestyle{terminalplain}{
  basicstyle=\ttfamily\small,
  breaklines=true,
  breakatwhitespace=false,
  breakautoindent=false,
  breakindent=0pt,
  numbers=none,
  showspaces=false,
  showstringspaces=false,
  showtabs=false,
  tabsize=4,
  frame=none,
  columns=fullflexible,
  keepspaces=true,
  upquote=true,
  literate={~}{{\textasciitilde}}1
           {\^}{{\textasciicircum}}1
           {\$}{{\textdollar}}1
           {\#}{{\#}}1
           {\\}{{\textbackslash}}1
}

% ===== tcbset 样式预设 =====
\tcbset{
  base dark/.style={
    colback=terminalbg,
    colframe=gray,
    coltext=terminalfg,
    coltitle=white,
    colbacktitle=black,
  },
  base light/.style={
    colback=white,
    colframe=gray,
    coltext=black,
    coltitle=black,
    colbacktitle=gray!20,
  },
  base common/.style={
    enhanced jigsaw,
    breakable,
    boxrule=0.5pt,
    arc=3pt,
    left=8pt,
    right=8pt,
    top=8pt,
    bottom=8pt,
    fonttitle=\ttfamily\small\bfseries,
    before skip=10pt,
    after skip=10pt,
    width=\textwidth,
  }
}

% ===== 有色终端环境 =====
% 用法:\begin{terminalcolored}{标题}{dark/light}
\newtcolorbox{terminalcolored}[2]{
  base common,
  base #2,
  fontupper=\ttfamily\small,
  title=#1,
  title after break=#1,
}

% ===== 无色终端环境(基于 listings)=====
% 用法:\begin{terminalplain}{标题}{dark/light}
\newtcblisting{terminalplain}[2]{
  base common,
  base #2,
  listing engine=listings,
  listing only,
  listing options={style=terminalplain,basicstyle=\ttfamily\small},
  title=#1,
  title after break=#1,
}

% ============================================================================
% 使用说明:
%
% 无色模式(推荐,自动处理特殊字符):
%   \begin{terminalplain}{Terminal}{dark}
%     (base) max@qmobile:~/Documents$ ls
%   \end{terminalplain}
%
% 有色模式(需要手动添加颜色命令):
%   \begin{terminalcolored}{Terminal}{dark}
%     (base) \textcolor{ansigreen}{max@qmobile} ls \\
%   \end{terminalcolored}
%
% 主题切换:
%   - 第二个参数使用 dark: 黑色背景,适合屏幕阅读
%   - 第二个参数使用 light: 白色背景,适合打印
%
% ============================================================================

\begin{document}

\section*{彩色终端环境示例(dark 主题)}

\begin{terminalcolored}{Terminal}{dark}
Script started on 2025-10-14 07:49:48+02:00 [TERM="xterm-256color" TTY="/dev/pts/0" COLUMNS="212" LINES="56"] \\
(base) \textcolor{lime}{max@qmobile}:\textcolor{blue}{\textasciitilde{}/Documents}\$ mkdir demo \&\& cd demo \\
(base) \textcolor{lime}{max@qmobile}:\textcolor{blue}{\textasciitilde{}/Documents/demo}\$ mkdir dir1 dir2 .hidden\_dir \\
(base) \textcolor{lime}{max@qmobile}:\textcolor{blue}{\textasciitilde{}/Documents/demo}\$ touch file1 file2 .hidden\_file exec\_file \\
(base) \textcolor{lime}{max@qmobile}:\textcolor{blue}{\textasciitilde{}/Documents/demo}\$ chmod +x exec\_file \\
(base) \textcolor{lime}{max@qmobile}:\textcolor{blue}{\textasciitilde{}/Documents/demo}\$ ls \\
\textcolor{blue}{dir1}  \ \textcolor{blue}{dir2}  \ \textcolor{lime}{exec\_file}  \ file1  \ file2 \\
(base) \textcolor{lime}{max@qmobile}:\textcolor{blue}{\textasciitilde{}/Documents/demo}\$ cd .. \&\& tree demo \\
\textcolor{blue}{demo} \\
├── \textcolor{blue}{dir1} \\
├── \textcolor{blue}{dir2} \\
├── \textcolor{lime}{exec\_file} \\
├── file1 \\
└── file2 \\

2 directories, 3 files \\
(base) \textcolor{lime}{max@qmobile}:\textcolor{blue}{\textasciitilde{}/Documents}\$ rm demo \# 测试报错 \\
rm: 无法删除 'demo': 是一个目录 \\
(base) \textcolor{lime}{max@qmobile}:\textcolor{blue}{\textasciitilde{}/Documents}\$ rm -r demo \\
(base) \textcolor{lime}{max@qmobile}:\textcolor{blue}{\textasciitilde{}/Documents}\$ exit \\
注销 \\

Script done on 2025-10-14 07:49:48+02:00 [COMMAND\_EXIT\_CODE="0"] \\
\end{terminalcolored}

\section*{无色终端环境示例(light 主题)}

\begin{terminalplain}{Terminal}{light}
Script started on 2025-10-14 07:46:41+02:00 [TERM="xterm-256color" TTY="/dev/pts/0" COLUMNS="212" LINES="56"]
(base) max@qmobile:~/Documents$ mkdir demo && cd demo
(base) max@qmobile:~/Documents/demo$ mkdir dir1 dir2 .hidden_dir
(base) max@qmobile:~/Documents/demo$ touch file1 file2 .hidden_file exec_file
(base) max@qmobile:~/Documents/demo$ chmod +x exec_file
(base) max@qmobile:~/Documents/demo$ ls
dir1  dir2  exec_file  file1  file2
(base) max@qmobile:~/Documents/demo$ cd .. && tree demo
demo
├── dir1
├── dir2
├── exec_file
├── file1
└── file2

2 directories, 3 files
(base) max@qmobile:~/Documents$ rm demo # 测试报错
rm: 无法删除 'demo': 是一个目录
(base) max@qmobile:~/Documents$ rm -r demo
(base) max@qmobile:~/Documents$ exit
注销

Script done on 2025-10-14 07:46:41+02:00 [COMMAND_EXIT_CODE="0"]

\end{terminalplain}

\end{document}
